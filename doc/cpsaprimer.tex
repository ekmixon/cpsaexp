\documentclass[12pt]{article}
\usepackage{amssymb}
\usepackage{amsmath}
\usepackage{url}
\usepackage{graphicx}
\usepackage{makeidx}
\usepackage{alltt}
\newcommand{\cpsa}{\textsc{cpsa}}
\newcommand{\version}{2.5.1}
\newcommand{\cpsacopying}{\begingroup
  \renewcommand{\thefootnote}{}\footnotetext{{\copyright} 2010 The
    MITRE Corporation.  Permission to copy without fee all or part of
    this material is granted provided that the copies are not made or
    distributed for direct commercial advantage, this copyright notice
    and the title of the publication and its date appear, and notice
    in given that copying is by permission of The MITRE
    Corporation.}\endgroup}
\newcommand{\pov}{\textsc{pov}}
\newcommand{\cow}{\textsc{cow}}
\newcommand{\cn}[1]{\ensuremath{\operatorname{\mathsf{#1}}}}
\newcommand{\dom}[1]{\ensuremath{\operatorname{\mathbf{#1}}}}
\newcommand{\fn}[1]{\ensuremath{\operatorname{\mathit{#1}}}}
\newcommand{\seq}[1]{\ensuremath{\langle#1\rangle}}
\newcommand{\enc}[2]{\ensuremath{\{\!|#1|\!\}_{#2}}}
\newcommand{\hash}[1]{\ensuremath{\##1}}
\newcommand{\comp}[2]{{\mathord{#1}}\circ{\mathord{#2}}}
\newcommand{\sembrack}[1]{[\![#1]\!]}
\newcommand{\semfn}[1]{\mathcal{#1}}
\newcommand{\sem}[2]{\semfn{#1}\sembrack{#2}}
\newcommand{\mcsu}{\ensuremath{\mathcal{C}}}
\newcommand{\probs}{\ensuremath{\mathcal{E}}}
\newcommand{\srt}[1]{\ensuremath{\mathsf{#1}}}
\newcommand{\eqq}{\stackrel{?}{=}}
\newcommand{\inbnd}{\mathord -}
\newcommand{\outbnd}{\mathord +}
\newcommand{\nat}{\mathbb{N}}
\newcommand{\all}[1]{\mathop{\forall#1}}
\newcommand{\some}[1]{\mathop{\exists#1}}
\newcommand{\baseterms}{\mathbb{B}}
\newcommand{\cons}{\mathbin{::}}
\newcommand{\ith}{\imath^\mathrm{th}}
\newcommand{\jth}{\jmath^\mathrm{th}}
\newcommand{\append}{\mathbin{{}^\smallfrown}}
\newcommand{\prefix}[2]{#1\dagger#2}
\newcommand{\orig}{\mathcal{O}}
\newcommand{\gain}{\mathcal{G}}
%\newcommand{\pow}[1]{\mathcal{P}(#1)}
%\newcommand{\pow}[1]{\wp(#1)}
\newcommand{\pow}[1]{2^{#1}}
\newcommand{\termat}{\mathbin{@}}
\newcommand{\idsigma}{\sigma_{\mathrm{id}}}
\newcommand{\idphi}{\phi_{\mathrm{id}}}
\newcommand{\kprec}[1]{\mathbin{\prec_{#1}}}
\newcommand{\terms}{\ensuremath{\mathcal{T}}}
\newcommand{\alg}[1]{\ensuremath{\mathfrak#1}}
\newcommand{\ops}[1]{\ensuremath{\mathbb#1}}
\newcommand{\bundle}{\ensuremath{\mathcal{B}}}
\newcommand{\der}[3]{\ensuremath{\dctx{#2}{#3}\vdash#1}}
\newcommand{\dctx}[2]{\ensuremath{#1:#2}}
\newcommand{\flow}[3]{\ensuremath{#1,#2\rhd#3}}
\newcommand{\infer}[2]{\begin{array}{c}#1\\\hline#2\end{array}}
\hyphenation{pre-skel-e-ton}
\newcommand{\homomorphism}[1]{\stackrel{#1}{\longmapsto}}
\newcommand{\reduction}[1]{\stackrel{#1}{\longrightarrow}}
\newcommand{\longtwoheadrightarrow}{\relax
\smash{\relbar\joinrel\twoheadrightarrow}\vphantom\rightarrow}
\newcommand{\setreduction}[1]{\stackrel{#1}{\longtwoheadrightarrow}}
\newcommand{\uniqat}{\mbox{\sf uniq-at}}
\newcommand{\strprec}{\mbox{\sf str-prec}}


\makeindex

\title{CPSA Primer}
\author{John D.~Ramsdell\qquad Joshua D.~Guttman\\
  The MITRE Corporation\\ CPSA Version \version}

\begin{document}
\maketitle
\cpsacopying

\tableofcontents

\listoffigures

\listoftables

\newpage

\section{Introduction}

\begin{sloppypar}
Analyzing a cryptographic protocol means finding out what security
properties---essentially, authentication and secrecy properties---are
true in all its possible executions.  Protocol analysis is hard
because an adversary can often manipulate the regular, law-abiding
participants.  The adversary may be able to manipulate the regular
participants into an unexpected execution, breaking a secrecy or
authentication property that the protocol was intended to ensure.
\end{sloppypar}

{\cpsa}, The Cryptographic Protocol Shapes Analyzer, is a software
tool.  Given a protocol definition and some assumptions about
executions, it attempts to produce descriptions of all possible
executions of the protocol compatible with the assumptions.
Naturally, there are infinitely many possible executions of a useful
protocol, since different participants can run it with varying
parameters, and the participants can run it repeatedly.

However, for many naturally occurring protocols, there are only
finitely many of these runs that are essentially different.  Indeed,
there are frequently very few, often just one or two, even in cases
where the protocol is flawed.  We call these essentially different
executions the \emph{shapes} of the protocol.  Authentication and
secrecy properties are easy to ``read off'' from the shapes, as are
attacks and anomalies.

The purpose of this document is to provide the background required to
make effective use of a {\cpsa} software distribution.  In particular,
the advice in Section~\ref{sec:advice} is essential reading.

The {\cpsa} program reads a sequence of problem descriptions, and
prints the steps it used to solve each problem.  Each input problem
contains some initial behavior, together with assumptions about some
uncompromised keys and freshly chosen values.  {\cpsa} discovers what
shapes are compatible with this problem description.  Normally, the
initial behavior is a local run of one participant, so that the
problem is to see what possible executions exist from that
participant's ``point of view.''  The analysis reveals what the other
participants must have done, given the first participant's view.

The shapes analysis is performed within a pure Dolev-Yao
model~\cite{DolevYao83}.  {\cpsa}'s search is based on a high-level
algorithm that was claimed to be complete, i.e.\@ every shape can in
fact be found in a finite number of
steps~\cite{DoghmiGuttmanThayer07,Guttman11}.  Further theoretical
work showed classes of executions that are not found by the algorithm,
however it also showed that every omitted execution requires an
unnatural interpretation of a protocol's roles.  Hence the algorithm
is complete relative to natural role semantics.
Appendix~\ref{sec:omitted executions} discusses omitted executions.

\section{Overview}

A {\cpsa} release includes several programs, an analyzer, and various
tools used to interpret the results.  The analyzer, \texttt{cpsa},
provides support for several algebras, one of which is the Basic
Crypto Algebra.  Programs that assist in the interpretation of results
are \texttt{cpsashapes} and \texttt{cpsagraph}.  The analyzer prints
the steps it used to solve each problem.  The \texttt{cpsashapes}
program extracts the shapes discovered by an analyzer run.  The
\texttt{cpsagraph} program graphs both forms of output using Scalable
Vector Graphics~(\textsc{svg}).  A standards-compliant browser such as
FireFox or Safari displays the generated diagrams.

The expected work flow follows.  An analysis problem is entered using
an ordinary text editor, preferably one with support for Lisp syntax.
Problem statement errors in the input are detected by running the
analyzer.  Many error reports are of the form that allow editors such
as Emacs to move its cursor to location of the problem.

There are two classes of problem statement errors: syntax and semantic
errors.  Correcting syntax errors is straightforward, but correcting
semantic errors requires an understanding of the core data structures.
Section~\ref{sec:semantic errors} describes their correction.

Once the problem statement errors have been eliminated, the analyzer
should produce useful output as a text document.  The text document
contains each step used to derive a shape from a problem statement.
It is common to filter the output using the \texttt{cpsashapes}
program, and look only at the computed shapes associated with each
problem statement.

The \texttt{cpsagraph} program is applied to the output to produce a
more readable, hyperlinked \textsc{xhtml} document that can be displayed
in a standards-compliant web browser.  The {\cpsa} User Guide contains
the up-to-date description of \texttt{cpsagraph} generated documents.
The guide is also the place to find command-line usage information for
all programs in a release.  The user guide is an \textsc{xhtml}
document delivered with the software.

The \texttt{cpsa} program uses S-expressions for both input and
output.  S-expression is an abbreviation for a Symbolic Expression of
Lisp fame, and is described in Appendix~\ref{sec:syntax reference}.

The input may optionally start with a \texttt{herald} form.  The form
contains a title for the run and an association list.  The association
list allows options normally specified on the command line to be
specified within an input file.  In the following example, the herald
form specifies a strand bound of 12 in a way that is equivalent to the
command line option \texttt{--bound=12}.

\begin{quote}
\begin{verbatim}
(herald Needham-Schroeder (bound 12))
\end{verbatim}
\end{quote}

The body of the input consists of two forms: protocol definitions and
initial behavior descriptions.  The exact details of both forms depend
on the message algebra specified by the protocol.  Protocols that
specify \texttt{basic} as their algebra get an implementation of the
Basic Crypto Algebra~(\textsc{bca}) described in the next section.  A
complete grammar for \texttt{cpsa} input with \textsc{bca} protocols
is displayed in Table~\ref{tab:syntax} on Page~\pageref{tab:syntax}.

\section{BCA Messages}

Each message exchanged in a protocol is represented by a term.  Terms
represent atomic values such as text objects, principal names, and
asymmetric and symmetric keys.  They also represent values composed
from other values via encryption and concatenation.

A sort system is used to classify terms. A {\cpsa} message algebra is
an order-sorted algebra~\cite{GoguenMeseguer92} with restrictions,
hence the use of the word sort instead of type.  Other aspects of
{\cpsa}'s use of order-sorted algebras is beyond the scope of this
paper.

The sorts that correspond to the atomic values are the base
sorts---for \textsc{bca} they are \fn{text}, \fn{data}, \fn{name},
\fn{akey}, and \fn{skey}.  The non-base sort is
\fn{mesg}.  Every term is of sort~\fn{mesg}, and every one of the
other sorts is a subsort of~\fn{mesg}.\index{sort}

The simplest term is a variable, which syntactically is a
\textsc{symbol} as described in Appendix~\ref{sec:syntax reference}.
Internally, each variable has a sort, so the sort of each variable in
the input must be declared in a \texttt{vars} form, such as:
$$\texttt{(vars (t text) (n name) (k akey))}.$$

Asymmetric keys come in pairs related by the \texttt{invk}
\index{operator}operator.  If~$t$ is a term of sort~\fn{akey}, so is
$\texttt{(invk~}t\texttt{)}$.  Furthermore,
$\texttt{(invk~(invk~}t\texttt{))}$ is equated with~$t$.  A name can
be used to identify an asymmetric key pair using the \texttt{pubk} and
\texttt{privk} operators, as in \texttt{(pubk~n)} and
\texttt{(privk~n)}---the latter is interpreted as
\texttt{(invk~(pubk~n))}.  A name may be associated with more than one
key using the binary form of \texttt{pubk}, where the first argument
is a quoted constant, as in
\texttt{(pubk~"sig"~n)} and
\texttt{(pubk~"enc"~n)}.
Two names can be used to identify a long
term symmetric key with the \texttt{ltk} operator.

Terms formed from base sorted variables, and the operators
\texttt{invk}, \texttt{pubk}, \texttt{privk}, and \texttt{ltk} are
called atoms, because each one represents an atomic value.  A key
property of an atom is that the receiver of an atom carried in a
message term cannot decompose the atom into parts.  For example, the
reception of a message that consists of the atom \texttt{(invk~k)}
does not allow its receiver to deduce~\texttt{k}.  Within this
document, S-expression syntax will often, for the sake of readability,
be replaced by the traditional notation for terms.  Thus
\texttt{(invk~k)} will be written as $K^{-1}$, and \texttt{(pubk~n)}
as $K_{N}$.

The terms in this algebra are freely generated from the atoms, tags,
concatenation, encryption, and hashing.  A tag is a quoted constant.
The concatenation of terms~$t$ and~$t'$ is~$t,t'$ in traditional infix
notation and \texttt{(cat $t$ $t'$)} in S-expression syntax.  The
comma operator is right associative and \texttt{(cat $a$ $b$ $c$ $d$)}
is equivalent to \texttt{(cat $a$ (cat $b$ (cat $c$ $d$)))}.
Given~$t$, the term to be protected, and key term~$k$, the encryption
of~$t$ using~$k$ is~$\enc{t}{k}$ in traditional notation and
\texttt{(enc $t$ $k$)} in S-expression syntax.  The term represents
asymmetric encryption when the key is of sort \dom{akey}, otherwise it
represents symmetric encryption.  Additionally, \texttt{(enc $a$ $b$
  $c$ $d$ $k$)} is equivalent to \texttt{(enc (cat $a$ $b$ $c$ $d$)
  $k$)}.  Given~$t$, the term to be hashed, its hash is $\hash{t}$ in
traditional notation and \texttt{(hash $t$)} in S-expression syntax.
{\cpsa} treats a hashed term as if it were an encryption in which the
term that is hashed is the encryption key.  As with encryption,
\texttt{(hash $a$ $b$ $c$ $d$)} is equivalent to \texttt{hash (cat $a$
  $b$ $c$ $d$))}.  Figure~\ref{fig:ns responder} on
Page~\pageref{fig:ns responder} contains examples of \textsc{bca}
message terms.  Also see \textsc{term} in Table~\ref{tab:syntax},
Appendix~\ref{sec:syntax reference}.

A message term \emph{carries}\index{carries} a subterm of the message
if the possession of the right set of keys allows the extraction of
the subterm.  The carries relation is the least relation such that
(1)~$t$ carries~$t$, (2)~$\enc{t_0}{t_1}$ carries~$t$ if~$t_0$
carries~$t$, and (3)~$t_0,t_1$ carries~$t$ if~$t_0$ or~$t_1$
carries~$t$.  As noted above, the message~$k^{-1}$ does not carry~$k$.
Also,~$\enc{t}{k}$ does not carry~$k$ unless (anomalously)~$t$
carries~$k$.

\begin{figure}
\begin{center}
\includegraphics{cpsadiagrams-0.mps}\hfil
\includegraphics{cpsadiagrams-1.mps}
\caption{Needham-Schroeder Initiator and Responder Roles}
\label{fig:ns roles}
\end{center}
\end{figure}

\section{Protocols}

A \index{protocol}\emph{protocol} defines the patterns of allowed
behavior for non-adversarial participants.  In other words, the
behavior of each participant must be an instance of some protocol
template, called a \index{role}\emph{role.}  Figure~\ref{fig:ns roles}
displays the roles that make up the Needham-Schroeder protocol.

In S-expression syntax, a protocol is a named set of roles and is
defined by the \texttt{defprotocol} form.  See \textsc{protocol} in
Table~\ref{tab:syntax}, Appendix~\ref{sec:syntax reference}.
\begin{center}
\begin{tabular}{l}
\verb|(defprotocol ns basic|\\
\verb|  (defrole init| \ldots\texttt{)}\\
\verb|  (defrole resp| \ldots\texttt{))}
\end{tabular}
\end{center}

The name of this protocol (\textsc{id}) is \texttt{ns}, and the second
identifier (\textsc{alg}) names the message algebra in use.  The
identifier for the Basic Crypto Algebra is \texttt{basic}.

A role has a name, a declared set of variables, and a trace that
provides a template for the behavior of its instances.  A trace is a
non-empty sequence of message events, either a message reception or a
transmission.  An inbound\index{inbound} message with term~$t$
is~$\inbnd t$ in text and \texttt{(recv $t$)} in S-expression syntax.
An outbound\index{outbound} term is~$\outbnd t$ in text and
\texttt{(send $t$)} in S-expression syntax.  The Needham-Schroeder
responder's role in S-expression syntax is in Figure~\ref{fig:ns
  responder}.

% A linear representation of the Needham-Schroeder responder's trace
% follows.
% $$\fn{resp}_t=\inbnd\enc{N_1, A}{K_B}\Rightarrow \outbnd\enc{N_1,
%  N_2}{K_A}\Rightarrow\inbnd \enc{N_2}{K_B}$$

\begin{figure}
\begin{quote}
\begin{verbatim}
(defrole resp (vars (b a name) (n2 n1 text))
  (trace (recv (enc n1 a (pubk b)))
         (send (enc n1 n2 (pubk a)))
         (recv (enc n2 (pubk b)))))
\end{verbatim}
\end{quote}
\caption{Needham-Schroeder Responder Role}
\label{fig:ns responder}
\end{figure}

Zero-based indexing\index{zero-based indexing}\index{indexing,
  zero-based} is used though out this document and in the source code
it describes.  Within the document, a finite sequence is a function
from an initial segment of the natural numbers.  Angle brackets are
used for sequence construction.  Thus $\seq{3,2,99} = \{0\mapsto 3,
1\mapsto 2, 2\mapsto 99\}$, and the responder's trace is
$\seq{\inbnd\enc{N_1, A}{K_B}, \outbnd\enc{N_1,
    N_2}{K_A},\inbnd\enc{N_2}{K_B}}.$ The length of a sequence~$x$ is
$|x|$.  \nocite{Dijkstra82}

A term \emph{originates}\index{origination} in a trace if it is
carried in some event and the first term in which it is
carried is an outbound term.  A term is
\emph{acquired}\index{acquired} by a trace if it first occurs in an
inbound term and is also carried by that term.

Some atoms in a role have special properties.  An atom may be declared
to be non-originating with the \texttt{non-orig} form or the
\texttt{pen-non-orig} form and uniquely originating with the
\texttt{uniq-orig} form.  The declarations make assertions about
instances of a role, assertions that will be defined after role
instantiation is explained.

Every variable that occurs in each term declared to be non-originating
must occur in some term in the trace, and the term must not be carried
by any term in the trace.  Every variable that occurs in each term
declared to be penetrator non-originating must occur in some term in
the trace, but the term may be carried by some term in a trace.  Each
term declared to be uniquely originating must originate in the trace.
Each variable of sort \fn{mesg} must be acquired in the trace.

\section{Executions}

A protocol analysis problem is specified by a skeleton.  Some
background information is presented before details of a problem
specification is given.

A skeleton describes a set of executions of a protocol.  They
specify the local behavior of participants and their interactions via
message-passing.  A definition of an execution is presented for the
case in which protocols declare no terms to be non-originating or
uniquely originating, and then is later amended.  The executions are a
representation of the strand space notion of bundle.

A strand\index{strand} represents a principal executing a single local
session of a protocol role.  The sequence of events that
describes the session is patterned after a prefix of its role's trace.
In the context of a protocol, a strand's trace is represented by an
\emph{instance}\index{instance}, a triple consisting of a role name, a
height, and a map from role variables to terms.  The length of the
described sequence is the instance's \emph{height}\index{height}, and
must be positive and no greater than the length of the associated
role's trace.  The map is the instance's
\index{environment}\emph{environment.}  The domain of the map is the
set of variables that occur in the events in the prefix of the
role's trace of the same length as the instance's height.  \iffalse An
instance is in $I=R\times\nat\times(X\rightarrow T)$.  \fi

An example of an instance of the Needham-Schroeder responder role
is:
\begin{center}
\texttt{(defstrand resp 2 (b a) (a b) (n1 n1) (n2 n2))}
\end{center}
The trace associated with this instance is:
$$\seq{\inbnd\enc{N_1, B}{K_A}, \outbnd\enc{N_1,
N_2}{K_B}}.$$

The \emph{position}\index{position} of a event in the
instance's trace is its index in the sequence, and in this example,
the position of $\inbnd\enc{N_1, B}{K_A}$ is zero.

In addition to a protocol, a component of an execution is a collection
of local sessions.  An instance cannot be used to identify one
session, because two principals may engage in the same pattern of
message passing.  Instead, a sequence of instances is used as a
component of an execution, and a \emph{strand} is represented by an
index that selects an instance from the sequence.  In other words,
each principal executing a local session is represented by a natural
number less than the length of the sequence of instances, and its
behavior is described by the instance found using its representation.

A \emph{node}\index{node} is a pair of natural numbers that identifies
an event within a sequence of instances.  The first integer is
the strand, and the second is the position of the event within
the strand's trace.  A node with an inbound term is a reception
node\index{reception node}, and one with an outbound term is a
\index{transmission node}transmission node.  Given a sequence~$i$, its
nodes are: $$\{(s,p)\mid s<|i|, p<\cn{height}(i(s))\}.$$

The remaining component of an execution is a binary relation between
transmission nodes and reception nodes, where each pair of nodes in
the relation agree on their message term.  The relation specifies
message transmissions between strands.  The nodes of the graph
associated with an execution are the nodes in the sequence of
instances, and the edges include the ones in the relation.
Additionally, the following strand succession edges\index{strand
  succession edges} are included: $$\{((s,p), (s, p+1))\mid s<|i|,
p+1<\cn{height}(i(s))\}.$$ Executions with cyclic graphs are omitted
from consideration, because they violate causality.  The transitive
irreflexive~$\prec$ is the transitive closure of the graph's edges,
and $n_0\prec n_1$ asserts that the message event at~$n_0$ precedes
the one at~$n_1$.

The node relation of an execution satisfies one additional property.
For each reception node in its sequence, there exists a unique
transmission node, related to it by the communication ordering.  In
other words, the relation must be a function.  Informally, this
property ensures that every message reception is accounted for by the
activity of a principal that is part of the execution.
Figure~\ref{fig:ns intended} shows the intended execution of the
Needham-Schroeder Protocol.  The node graph of an execution is a
bundle.

A set of runs of the protocol is associated with each execution.  A
variable is associated with an execution if the variable occurs in the
range of an environment in some instance.  To derive a run from an
execution, a substitution that maps each variable associated with the
execution to a ground term is applied to the execution.  {\cpsa}
message algebras do not contain the constants required to name all the
ground terms.  For most sorts, the identity of a particular ground
term is irrelevant to the analysis.  Message tags are the only
exception to this rule.  For this reason, {\cpsa} message algebras are
free algebras, but not initial algebras.

Strands in executions represent both adversarial and non-adversarial
behaviors.  A strand that is an instance of a protocol role is
non-adversarial, and is called \index{regular}\emph{regular.}  A
strand that represents adversarial behavior is called a
\emph{penetrator}\index{penetrator} strand.

\begin{figure}
\begin{center}
\includegraphics{cpsadiagrams-2.mps}
\end{center}
\begin{quote}
\begin{verbatim}
(vars (n1 n2 text) (a b name))
(defstrand init 3 (n1 n1) (n2 n2) (a a) (b b))
(defstrand resp 3 (n2 n2) (n1 n1) (b b) (a a))
(precedes ((0 0) (1 0)) ((1 1) (0 1)) ((0 2) (1 2)))
\end{verbatim}
\end{quote}
\caption{Needham-Schroeder Intended Run}
\label{fig:ns intended}
\end{figure}

The roles that define adversary behavior codify the basic abilities
that make up the Dolev-Yao model.  They include transmitting an atom
such as a name or a key; transmitting a tag; transmitting an encrypted
message after receiving its plain text and the key; and transmitting a
plain text after receiving ciphertext and its decryption key.  The
adversary can also concatenate two messages, or separate the pieces of
a concatenated message.  Since a penetrator strand that encrypts or
decrypts must receive the key as one of its inputs, keys used by the
adversary---compromised keys---have always been transmitted by some
participant.

Figure~\ref{fig:ns penetrated} shows a penetrated execution of the
Needham-Schroeder Protocol.  Strand space theory would decompose the
penetrator behavior into multiple strands; instead the description of
the penetrator has been simplified by the use of an artificially
constructed role~\fn{pen}.

A \index{non-origination}\emph{non-originating term} is an atom
that is carried by no message, and it or its inverse is the key of an
encryption in some message.  Each non-originating term is a key that
is not compromised, as the penetrator has no access to the key.

A \index{penetrator non-origination}\emph{penetrator non-originating
  term} is an atom that is forbidden from being originated on a
penetrator strand in an execution.

A \index{unique origination}\emph{uniquely originating term} is an
atom that originates on exactly one strand in the execution.  The
correct behavior of some protocols depends on the fact that its
executions include only ones in which some terms are uniquely
originating---each term's provenance is one node.  Implementations of
these protocols can ensure unique origination of a term by freshly
generating a nonce for the component of the message it represents.

Occasionally, protocol roles designate some terms to be
non-originating or uniquely originating.  Consider the prefix of the
trace of a role associated with some instance.  When a uniquely
originating role term is carried in the prefix, the instance's
environment maps it to a term that must originate in the instance's
strand.  Furthermore, when the variables in a non-originating role
term occur in the prefix, the instance's environment maps that term to
one that must not be carried by any message term in the execution.
The mapping of a non-originating role term can be conditioned on the
height of the instance.  A role non-origination assumption of the form
\texttt{(3 a)} asserts that \texttt{a} will not be mapped into an
instance unless its height is at least three.  Penetrator
non-originating atoms are similar.  Section~\ref{sec:advice} provides
advice on when to add non-origination or unique origination
assumptions to roles.

\begin{figure}
\begin{center}
\includegraphics{cpsadiagrams-3.mps}
\caption{Needham-Schroeder Penetrated}
\label{fig:ns penetrated}
\end{center}
\end{figure}

\section{Skeletons}\index{skeleton}

A \emph{skeleton} represents regular behavior that might make up part
of an execution.  The components of a skeleton are similar to an
execution.  The components of a skeleton include a protocol, a
sequence of instances, and a binary relation between transmission
nodes and reception nodes within the sequence.  Unlike an execution,
the strands in a skeleton specify only regular behavior.  Furthermore,
the pair of nodes in the relation need not agree on their message
term.  Two nodes are related if the transmitting node precedes the
reception node, as an execution it represents may include nodes
between the related transmission and reception nodes.

The final three additional components of a skeleton are a set of
non-originating terms, a set of penetrator non-originating terms, and
a set of uniquely originating terms.  To be a skeleton, each uniquely
originating term must originate in at most one strand in the skeleton,
and each non-originating term must never be carried by some event in
the skeleton and every variable that occurs in the term must occur in
some event.  For a penetrator non-originating term, is suffices that
every variable that occurs in the term must occur in some event.
Furthermore, for each uniquely originating term that originates in the
skeleton, the node relation must ensure that reception nodes that
carry the term follow the node of its origination.
Figure~\ref{fig:skeleton} shows the components in the S-expression
representation of a skeleton.

\begin{figure}
\begin{quote}
\begin{alltt}
(defskeleton \textsc{protocol} \textsc{variables}
   \ldots                ; Instance sequence
   (precedes \ldots)     ; Node orderings
   (non-orig \ldots)     ; Non-originating terms
   (pen-non-orig \ldots) ; Penetrator non-originating terms
   (uniq-orig \ldots))   ; Uniquely originating terms
\end{alltt}
\end{quote}
\caption{Components of Skeletons}\label{fig:skeleton}
\end{figure}

Two skeletons are equivalent if there is a permutation of the strands
in one skeleton that when applied to its sequence and its node
relation, produces the other skeleton.  Two skeletons are also
equivalent if they differ only by a systematic, sort-preserving
renaming of their variables, excluding variables that occur in the
domains of environments---the role variables.  The trace of a strand
is used for the comparison, so role names need not match.

A skeleton is normalized by performing a transitive reduction on its
node relation.  The transitive reduction\index{transitive reduction}
of an ordering is the minimal ordering such that both orderings have
the same transitive closure.  Here, communication orderings implied by
transitive closure are removed.  Two skeletons are equivalent if their
normalized forms are equivalent.

One special skeleton is associated with each execution.  It summarizes
the regular behavior of the execution.  It is derived from the
execution by enriching its node relation to contain all node orderings
implied by transitive closure, deleting all strands and nodes that
refer to penetrator behavior, and then performing the transitive
reduction on the resulting node relation.  The set of uniquely
originating terms is the set of terms that originate on exactly one
strand in the execution, and are carried in a term of a regular
strand.  The set of non-originating terms is the union of two sets.
One set contains each term that is used as an encryption or decryption
key in some term in the execution, but is not carried by any term.
The other set contains the terms specified by non-origination
assumptions in roles.  If a realized skeleton instance maps all of the
variables that occur in one of its non-originating role terms, the
mapped term is a member of the skeleton's set of non-originating
terms.  A skeleton is \emph{realized}\index{realized skeleton} if it
summarizes the behavior of some execution.

Skeletons are a central concept in the {\cpsa} algorithm because they
capture the notion that the details of penetrator behavior are
irrelevant to the analysis.  From the perspective of regular behavior,
all that is needed is a description of the messages that can be
derived by the penetrator at a given reception node of a regular
strand.  In a realized skeleton, some combination of penetrator
behavior and regular behavior derives the message at every reception
node.  The skeleton's node relation specifies the transmission nodes
that provide messages available to the penetrator for message
derivations.  The rules for message derivation are algebra specific.
If the message derivation rules imply a message at a reception node in a
skeleton is derivable, the node is~\emph{realized.}

A skeleton with an unrealized node might be related to another
skeleton with additional regular behavior that makes the original node
realized.  A skeleton with additional regular behavior is called a
refinement.

A skeleton~$A$ structurally refines~$B$ if the transitive closure of the
graph associated with~$B$ is a subgraph of the one associated
with~$A$, the event at corresponding nodes agree, the set of
uniquely originating terms of~$B$ is a subset of the ones in~$A$, the
set of non-originating terms of~$B$ is a subset of the ones in~$A$,
and a uniquely originating term that originates in~$B$ originates at
the corresponding node in~$A$.  Penetrator non-originating assumptions
are similar to non-originating assumptions.

A skeleton~$A$ message refines~$B$ if~$A$ and~$B$ agree on all but the
terms in the range of each strand's environment and its
non-originating and uniquely originating terms, and there is a
substitution that maps each term in~$B$ to its related term in~$A$.
Similar to structural refinement, a uniquely originating term that
originates in~$B$ originates at the same node in the image of~$B$.

A skeleton~$A$ refines\index{refinement}~$B$ if~$A$ is equivalent to a
skeleton that structurally or message refines~$B$.\footnote{A
  skeleton~$A$ refines~$B$ if there is a homomorphism from~$B$ to~$A$
  as defined in~\cite{DoghmiGuttmanThayer07}.  The implementation
  avoids the complexities of directly representing homomorphisms by
  composing structural and message refinement with equivalence checks,
  as is done in this chapter.}  Each skeleton describes the realized
skeletons that refine it.  A skeleton is \emph{dead} if no realized
skeleton refines it.  A diagram of a skeleton is in Figure~\ref{fig:ns
  shape}.

\begin{figure}
\begin{center}
\includegraphics{cpsadiagrams-4.mps}
\caption{Needham-Schroeder Shape ($K^{-1}_A$ uncompromised, $N_2$ fresh)}
\label{fig:ns shape}
\end{center}
\end{figure}

The {\cpsa} algorithm computes realized skeletons from unrealized
skeletons by identifying an unrealized node, and computing the
skeletons that refine the unrealized skeleton by making the target
node realized.

\section{Listeners}\index{listeners}

In addition to the roles specified in a protocol, for each term~$t$, a
regular strand may be an instance of the listener role with the trace
$\fn{lsn}(t) = \seq{\inbnd t, \outbnd t}$.  There are no non-originating or
uniquely originating terms associated with a listener role.

A listener strand is used in a skeleton to assert that an atom~$t$ is
available on its own to the adversary, unprotected by encryption.  For
example, to test if the protocol keeps a term~$t$ from the adversary,
one adds a strand that listens for~$t$.  The term is protected if the
resulting skeleton is dead.  Otherwise, the {\cpsa} analyzer program
will find a refined realized skeleton that shows how the adversary
accesses~$t$.

A listener instance in S-expression syntax follows.  See
\textsc{strand} in Table~\ref{tab:syntax}, Appendix~\ref{sec:syntax
  reference}.
\begin{center}
\texttt{(deflistener }\textsc{term}\texttt{)}
\end{center}
The {\cpsa} programs generate the associated listener role and hide
it on output.  Listener role names are absent in all forms of output,
one indication that a strand is an instance of a listener role.

\section{Authentication Tests}\label{sec:authentication tests}
\index{nonce test}\index{encryption test}

Authentication tests~\cite{GuttmanThayer02} guide the search for
skeletons that refine one with an unrealized node into ones in which
it is realized.  There are two types of authentication tests, nonce
and encryption tests.  In both cases, an unrealized node is selected,
called the \index{test node}\emph{test node.}  A term carried by the
inbound message at the test node is identified as the \index{critical
  term}\emph{critical term.}  A critical term is one that occurs in a
message context, the construction of which cannot be explained by the
regular behavior in the current skeleton or by penetrator behavior.
An authentication test determines the additional regular behavior
required to refine a skeleton into ones in which the test node is
realized.

The critical term in a nonce test is a uniquely originating term, the
nonce.  It is freshly generated by one regular participant in each run
of the protocol.  A nonce is unguessable by both regular and
adversarial participants except when it is received in an unprotected
context.

A reception node's \index{outbound predecessors}\emph{outbound
  predecessors} is the set of messages sent by transmission nodes that
precede the reception node, as given by the transitive closure of its
skeleton's node ordering relation.  Suppose every occurrence of the
nonce in the outbound predecessors of the test node is within a
context protected by encryption, but the critical term in the test
node occurs outside of all of those encryptions.  Clearly, another
participant was able to decrypt one of the test node's outbound
predecessors.  The set of encryptions that protects a critical term
in a test node's outbound predecessors is called its \index{escape
  set}\emph{escape set.}

There are three ways to refine a skeleton to account for the
decryption: regular augmentation, listener augmentation, and
contraction.  For regular augmentation, an instance of a protocol role
is added to the skeleton.  The final transmission node of the strand
is called a \emph{transforming node.}  The strand is selected for
augmentation because relative to the test node's outbound
predecessors, the transforming node's message shows the strand
performed the decryption.

Sometimes, a regular augmentation is immediately followed by merging
the new strand into another.  This is called a displacement.  It
occurs when simply adding the strand creates a redundancy.

The penetrator can expose the critical term if it has access to any
of the decryption keys used to protect it in the escape set.  For
listener augmentation, an instance of a listener role listening to one
decryption key used to protect the critical term is added, and the
skeleton's communication ordering relation is updated to record the
fact that the listener's transmission node precedes the test node.  If
the resulting skeleton is not dead, the decryption can be explained by
penetrator behavior.

For a contraction, a message refining substitution is found that
equates two or more atoms.  Sometimes, the key used to protect the
critical term can be equated with one used in previous messages,
thus vacuously explaining the decryption of the critical term.

The critical term in an encryption test is an encryption.  When the
encryption key is unavailable, the encryption is unguessable.
Whenever the unavailability of the encryption key can be established,
the methods used to refine skeletons with nonce tests apply.
Additionally, an instance of a listener role listening for the
critical term's encryption key is added.  If the resulting skeleton
is not dead, the encryption can be explained by penetrator behavior.

\section{Generalization}

Repeated use of authentication tests either produce realized skeletons
or show that a skeleton is dead.  The next step in the algorithm is to
make each realized skeleton into a shape, using the process of
generalization.

Realized skeleton~$A$ \emph{generalizes}\index{generalization}
realized skeleton~$B$ if~$A$ refines the origin problem specification,
and $B$ refines~$A$.  Furthermore,~$A$ may not combine strands in~$B$.
The \emph{shape}\index{shape} associated with a realized skeleton is
its maximally generalized realized skeleton.  The shapes of a protocol
capture all the essentially different executions possible for the
protocol consistent with the initial behavior specification.
Figure~\ref{fig:ns shape} shows the shape associated with the
Needham-Schroeder Protocol from the point of view of a responder
strand.

A different fixed set of operations is used to transform a realized
skeleton into a shape: deletion, weakening, separation, and
forgetting.  If an operation succeeds, it produces a more general
realized skeleton, one that is not equivalent to the starting
skeleton.  If no operations succeeds, the skeleton is a shape.

For deletion, a node and all nodes that follow it in a strand are
deleted, and the resulting skeleton is checked to see if it
generalized the starting skeleton.  The operation is tried for each
node in the starting skeleton until there is a success.

If deletion fails, a skeleton is weakened by deleting one element of
the communication ordering, then checking the result to see if it
generalized the starting skeleton.  The operation is tried for each
communication ordering in the starting skeleton until there is a success.

If weakening fails, origination assumption forgetting is tried by
deleting each term in the non-originating set that is not
specified by a role.  This is followed by deleting each term in the
uniquely originating set that is not specified by a role.

If origination assumption forgetting fails, variable separation is
tried.  Sometimes a more general skeleton can be found by replacing
some occurrences of one variable by a fresh variable.  To separate a
variable, the collection of places at which the variable occurs in the
range of all environments is generated, and a fresh variable is
substituted for the variable at a subset of these places.  All
possibilities are tried until a more general skeleton is found.

Sometimes a shape is derived from another shape by collapsing two
strands in a shape.  Collapsing might produce an unrealized skeleton,
so authentication tests apply.

\section{Skeletons}\index{skeleton}\label{sec:skeletons}

With this background, the \texttt{defskeleton} form in
Table~\ref{tab:syntax}, Appendix~\ref{sec:syntax reference} is
explained.  The key object in {\cpsa} input and output is a skeleton,
but an object with weaker properties is allowed for the initial
problem statement.  A \index{preskeleton}\emph{preskeleton} is a
skeleton except that terms in the uniquely originating set may
originate in more than one strand.  Furthermore, the node relation of
a preskeleton need not imply that a node that carries a uniquely
originating term is after the node of its origination.  A preskeleton
that cannot be immediately converted into a skeleton is erroneous, and
an error message is issued.

Referring to \textsc{skeleton} in Table~\ref{tab:syntax}, the
\textsc{id} in the skeleton form names a protocol.  It refers to
the most recent protocol definition of that name which precedes the
skeleton form.  The \textsc{id} in the strand form names a role.
The integer in the strand form gives the height of the strand.  The
sequence of pairs of terms in the strand form specify an environment
used to construct the messages in a strand from its role's trace.  The
first term is interpreted using the role's variables and the second
term uses the skeleton's variables.  The environment used to
produce the strand's trace is derived by matching the second term
using the first term as a pattern.

The \texttt{precedes} form specifies members of the node relation.
The first integer in a node identifies the strand using the order in
which strands are defined in the \texttt{defskeleton} form.

A variable may occur in more then one role within a protocol.  The
reader performs a renaming so as to ensure these occurrences do not
overlap.  Furthermore, the maplets used to specify a strand need not
specify how to map every role variable.  The reader inserts missing
mappings, and renames every skeleton variable that also occurs in a
role of its protocol.  The sort of every skeleton variable that occurs
in the \texttt{non-orig}, \texttt{pen-non-orig}, or \texttt{uniq-orig}
list or in a maplet must be declared, using the \texttt{vars} form.

The \textsc{prot-alist}, \textsc{role-alist}, and \textsc{skel-alist}
productions are Lisp style association lists, that is, lists of
key-value pairs, where every key is a symbol.  Key-value pairs with
unrecognized keys are ignored, and are available for use by other
tools.  On output, unrecognized key-value pairs are preserved when
printing protocols, but elided when printing skeletons, with the
exception of the \texttt{comment} key.

\subsection{Semantic Errors in the Input}\label{sec:semantic errors}

The error messages generated for syntax errors are informative,
however the ones generated for semantic errors are less so.  A role
might be rejected because it is not \index{well-formed
  role}well-formed.  A role is not well-formed if (1) there is a term
declared to be uniquely originating that does not originate in the
trace, (2) there is a term declared to be non-originating that is
carried by some term in the trace, a variable occurs in the term that
does not occur in the trace, or the declaration of the term included a
height, and a variable occurs in the term that does not occur in the
prefix of the trace of the given height, or (3) a variable of sort
\dom{mesg} is not acquired in the trace.  The error message might not
indicate which condition caused the rejection.

Similarly, a skeleton might be rejected because it is not
\index{well-formed skeleton}well-formed.  A skeleton is not
well-formed if (1) the first node in a node pair refers to an inbound
term, or the second node refers to an outbound term, (2) the node
ordering contains cycles, (3) a term declared to be uniquely
originating, is not carried by any term, (4) an instance maps a
uniquely originating role term to a term that does not originate in
the instance's strand, or (5) a term declared to be non-originating is
carried by a term in some strand, or a variable occurs in the term
that does not occur in any strand.  Once again, the error message might
not indicate which condition caused the rejection.

\subsection{Needham-Schroeder Input}

\begin{sloppypar}
This section contains the verbatim input of the running example used
throughout this paper.  The use of an editor that pretty-prints
S-expressions is recommended.
\end{sloppypar}

\begin{verbatim}
;;; Hey Emacs, use -*- mode:scheme -*-
(herald "Needham-Schroeder Public-Key Protocol"
        (comment "This protocol contains a man-in-the-middle"
                 "attack discovered by Galvin Lowe."))
\end{verbatim}

An S-expression version of Figure~\ref{fig:ns roles} follows.

\begin{verbatim}
(defprotocol ns basic
  (defrole init
    (vars (a b name) (n1 n2 text))
    (trace
     (send (enc n1 a (pubk b)))
     (recv (enc n1 n2 (pubk a)))
     (send (enc n2 (pubk b)))))
  (defrole resp
    (vars (b a name) (n2 n1 text))
    (trace
     (recv (enc n1 a (pubk b)))
     (send (enc n1 n2 (pubk a)))
     (recv (enc n2 (pubk b))))))
\end{verbatim}

The protocol is analyzed from the point of view of a complete run of
one instance of an initiator role.

\begin{verbatim}
(defskeleton ns
  (vars (a b name) (n1 text))
  (defstrand init 3 (a a) (b b) (n1 n1))
  (non-orig (privk b) (privk a))
  (uniq-orig n1))
\end{verbatim}

\section{Output}\label{sec:output}

The {\cpsa} output format has been designed so that it can be reused
as input.  All skeletons in the output are normalized skeletons,
with the possible exception of an initial preskeleton, the one used to
state a problem.  If an initial preskeleton cannot be converted into a
skeleton, an error is immediately signaled.

For each skeleton, a {\cpsa} analyzer computes a set of skeletons that
refine it using a fixed set of operations based on authentication
tests.  The immediate descendants of a skeleton are called its
\index{cohort}\emph{cohort.}  A member of the cohort that is
equivalent to a previously seen skeleton is replaced by that skeleton.
Thus the skeletons in an analysis form a directed acyclic graph.  All
but one skeleton has a single parent, and one skeleton can be a member
of several cohorts.

The operations above either produce realized skeletons or show that a
skeleton is dead.  The next step in the algorithm is to make each
realized skeleton into a shape, using the process of generalization.
Although the set of shapes is in some sense the answer to the problem,
an understanding of the operations used to generate the shapes can be
very informative.  The remainder of this section describes the
annotations in the output that allow for an understanding of each step
of the analysis.

On output, key-value pairs are added to each skeleton's association
list, \textsc{skel-alist}.  Every skeleton in the output is labeled
with a unique identifier with
\index{label}\texttt{(label~}\textsc{integer}\texttt{)}.  A skeleton
has \texttt{(parent~}\textsc{integer}\texttt{)} if it is a member of
the cohort of the identified parent.  A skeleton has
\texttt{(seen~}\textsc{integer+}\texttt{)} when members of its cohort
are equivalent to previously seen skeletons.  A skeleton lists its
unrealized nodes with
\texttt{(unrealized~}\textsc{node$\ast$}\texttt{)}.  The traces
associated with each strand is given by the
\texttt{(traces~}\ldots\texttt{)} form.

\begin{figure}
\begin{verbatim}
(defskeleton ns
  (vars (n1 n2 n2-0 text) (a b name))
  (defstrand init 3 (n1 n1) (n2 n2) (a a) (b b))
  (defstrand resp 2 (n2 n2-0) (n1 n1) (b b) (a a))
  (precedes ((0 0) (1 0)) ((1 1) (0 1)))
  (non-orig (privk a) (privk b))
  (uniq-orig n1)
  (operation nonce-test (added-strand resp 2) n1 (0 1)
    (enc n1 a (pubk b)))
  (traces
    ((send (enc n1 a (pubk b))) (recv (enc n1 n2 (pubk a)))
      (send (enc n2 (pubk b))))
    ((recv (enc n1 a (pubk b)))
      (send (enc n1 n2-0 (pubk a)))))
  (label 1)
  (parent 0)
  (unrealized (0 1))
  (comment "1 in cohort - 1 not yet seen"))
\end{verbatim}
\caption{Annotated {\cpsa} Output}\label{fig:output}
\end{figure}

Figure~\ref{fig:output} shows a skeleton generated during an analysis
of the Needham-Schroeder Protocol from the point of view of an
initiator strand.  It is labeled as~1.  It's parent is labeled~0.  It
has one child, and that child has not been seen before.  It is
unrealized.

\label{sec:operation}
\begin{sloppypar}
The operation used to derive a skeleton is recorded with
\index{operation}\texttt{(operation~}\textsc{test kind term node
  term$\ast$}\texttt{)}, where \textsc{test} is the authentication
test \texttt{encryption-test} or \texttt{nonce-test}, \textsc{kind} is
\texttt{(added-strand} \textsc{id} \textsc{integer}\texttt{)},
\texttt{(displaced} \textsc{integer} \textsc{integer} \textsc{id} \textsc{integer}\texttt{)},
\texttt{(contracted} \textsc{maplet$\ast$}\texttt{)}, or
\texttt{(added-listener} \textsc{term}\texttt{)}, \textsc{term} is the
critical term, \textsc{node} in the test node, and the remaining terms
specify the escape set. When the operation kind is added strand, the
instance's role name and height are provided.
For kind displaced, the first number is the strand being merged, the
second number is the strand it was merged into, and the remaining give
the new instance's role name and height.
For kind added-listener,
a term is provided. For kind contracted, the substitution is
provided. When a substitution refers to a variable not in the
skeleton, its name is unpredictable. For generalization, the operation
is recorded as
\texttt{(operation~generalization~}\textsc{method}\texttt{)}, where
\textsc{method} is one of \texttt{deleted} \textsc{node},
\texttt{weakened} \textsc{node-pair}, \texttt{separated}
\textsc{term}, or \texttt{forgot} \textsc{term}.  Shapes can be
collapsed leading to new shapes.  For shape collapsing, the operation
is recorded as \texttt{(operation~collapsed} \textsc{integer}
\textsc{integer}\texttt{)}, where the two \textsc{integer}s identify
the strands merged.
\end{sloppypar}

The skeleton in Figure~\ref{fig:output} was generated as a result of a
nonce test, by augmenting the starting skeleton with a responder
strand of length two.  The critical term is \texttt{n1}, the test
node is \texttt{(0~1)}, and the escape set has one element.

When the operation kind is added strand, it is possible that the
number of strands in the skeleton and its parent are the same.  In
this case, {\cpsa} has found a way to produce a more concise
representation of the skeleton by merging two strands.

\subsection{Needham-Schroeder Output}

% Use --margin=60 to generate output

This section contains the verbatim output of the running example used
throughout this paper.  A run starts by displaying the program's
version number.

\begin{verbatim}
(herald "Needham-Schroeder Public-Key Protocol"
  (comment "This protocol contains a man-in-the-middle"
    "attack discovered by Galvin Lowe."))
\end{verbatim}

\begin{flushleft}
\texttt{(comment "CPSA {\version}")}\\
\texttt{(comment "All input read")}
\end{flushleft}

An S-expression version of Figure~\ref{fig:ns roles} follows.

\begin{verbatim}
(defprotocol ns basic
  (defrole init
    (vars (a b name) (n1 n2 text))
    (trace (send (enc n1 a (pubk b)))
      (recv (enc n1 n2 (pubk a))) (send (enc n2 (pubk b)))))
  (defrole resp
    (vars (b a name) (n2 n1 text))
    (trace (recv (enc n1 a (pubk b)))
      (send (enc n1 n2 (pubk a)))
      (recv (enc n2 (pubk b))))))
\end{verbatim}

The protocol is analyzed from the point of view of a complete run of
one instance of an initiator role.

\begin{verbatim}
(defskeleton ns
  (vars (n1 n2 text) (a b name))
  (defstrand init 3 (n1 n1) (n2 n2) (a a) (b b))
  (non-orig (privk a) (privk b))
  (uniq-orig n1)
  (traces
    ((send (enc n1 a (pubk b))) (recv (enc n1 n2 (pubk a)))
      (send (enc n2 (pubk b)))))
  (label 0)
  (unrealized (0 1))
  (comment "1 in cohort - 1 not yet seen"))
\end{verbatim}

A nonce test justifies adding an instance of part of a reponder role.

\begin{verbatim}
(defskeleton ns
  (vars (n1 n2 n2-0 text) (a b name))
  (defstrand init 3 (n1 n1) (n2 n2) (a a) (b b))
  (defstrand resp 2 (n2 n2-0) (n1 n1) (b b) (a a))
  (precedes ((0 0) (1 0)) ((1 1) (0 1)))
  (non-orig (privk a) (privk b))
  (uniq-orig n1)
  (operation nonce-test (added-strand resp 2) n1 (0 1)
    (enc n1 a (pubk b)))
  (traces
    ((send (enc n1 a (pubk b))) (recv (enc n1 n2 (pubk a)))
      (send (enc n2 (pubk b))))
    ((recv (enc n1 a (pubk b)))
      (send (enc n1 n2-0 (pubk a)))))
  (label 1)
  (parent 0)
  (unrealized (0 1))
  (comment "1 in cohort - 1 not yet seen"))
\end{verbatim}

A nonce test justifies a contraction that produces the one and only
shape.  The shape is also displayed in Figure~\ref{fig:ns init pov}.

\begin{figure}
\begin{center}
\includegraphics{cpsadiagrams-5.mps}
\caption{Needham-Schroeder Shape (Initiator Point of View)}
\label{fig:ns init pov}
\end{center}
\end{figure}

\begin{verbatim}
(defskeleton ns
  (vars (n1 n2 text) (a b name))
  (defstrand init 3 (n1 n1) (n2 n2) (a a) (b b))
  (defstrand resp 2 (n2 n2) (n1 n1) (b b) (a a))
  (precedes ((0 0) (1 0)) ((1 1) (0 1)))
  (non-orig (privk a) (privk b))
  (uniq-orig n1)
  (operation nonce-test (contracted (n2-0 n2)) n1 (0 1)
    (enc n1 n2 (pubk a)) (enc n1 a (pubk b)))
  (traces
    ((send (enc n1 a (pubk b))) (recv (enc n1 n2 (pubk a)))
      (send (enc n2 (pubk b))))
    ((recv (enc n1 a (pubk b)))
      (send (enc n1 n2 (pubk a)))))
  (label 2)
  (parent 1)
  (unrealized)
  (shape))
\end{verbatim}

The following phrase means {\cpsa} is finished with this problem---its
exhausted its to do list.

\begin{verbatim}
(comment "Nothing left to do")
\end{verbatim}

\section{First-Order Logic and Security Goals}\label{sec:security goals}

Another way to specify a problem statement is with a security goal. A
security goal a sentence in first-order logic.  It asserts that if
some properties hold for a skeleton, then some other properties must
hold for all shapes computed by CPSA starting with the skeleton.
Security goals can be used to formally state authentication and
secrecy goals of a protocol.

The \texttt{defgoal} form is used to pose an analysis problem with a
sentence instead of a skeleton.  CPSA extracts a point of view
skeleton from the antecedent of the formula and then analyzes it.
When it prints a shape, it checks to see if the shape satisfies the
conclusion of the security goal.  See~\cite{cpsagoals09} for details.
The grammar of security goals is in Table~\ref{tab:goal}.

\section{Macros}\label{sec:macros}

After reading the input, {\cpsa} expands macros before in analyzing
the results. A macro definition is a top-level form.\index{macros}

\begin{quote}\scshape
(\texttt{\textup{defmacro}} (name arg${}^\ast$) body)
\end{quote}

The {\cpsa} program expands all calls to macros in non-macro defining
S-expressions using the macros that have been defined previously. A
macro definition can be used to expand a call if the first element of
a list matches the name of the macro, and the length of the remaining
elements in the list matches the length of the macro's argument
list. When two macros definitions are applicable, the last definition
takes precedence. The {\cpsa} program omits macro definitions from its
output.

After expanding a list, elements of the list of the form
\verb|(^ |$\cdots$\verb| )| are spliced into the output.  Thus
\verb|(a ^(b c) d)| becomes \verb|(a b c d)| after macro expansion.

\section{Includes}\label{sec:include}

While performing macro expansion, {\cpsa} includes other source files
with the top-level include form,\index{includes}
\begin{quote}\scshape
(\texttt{\textup{include}} file)
\end{quote}
where \texttt{\scshape file} is a string.

\section{Advice}\label{sec:advice}

This section contains advice derived from using {\cpsa}.  When
specifying {\cpsa} input, one must decide when to specify terms as
uniquely originating or non-originating in a role, and when to
specifying them in the initial skeleton.  If a fresh value is
generated by all programs that implement a role in a protocol, the
term that represents the fresh value should be assumed to be uniquely
originating in the role.  Otherwise, unique origination assumptions
should be specified in the initial skeleton.

Adding non-origination assumptions to a role can lead to an
excessively weak protocol analysis, i.e.\ an analysis relative to an
unrealistically narrow assumption.  Placing non-origination
assumptions in the initial skeleton is preferred whenever possible.

The Basic Crypto Algebra has two text-like sorts, \dom{text} and
\dom{data}.  For some protocols, message fields that carry uniquely
originating data cannot be carried by other text-like fields.  To
ensure {\cpsa} does not explore skeletons in which uniquely
originating data is carried in fields with predictable values, the
convention is to use the sort \dom{data} for uniquely originating
data, and sort \dom{text} for the other fields.

When looking at the output, try extracting the shapes first.  If the
shapes only version of the output does not answer your questions, try
studying the output that contains intermediate skeletons.

When {\cpsa} generates an unexpected intermediate skeleton, study its
operation field (see Page~\ref{sec:operation}).  Usually, the
unexpected intermediate skeletons of interest have been generated as a
result of an authentication test.  Section~\ref{sec:authentication
  tests} explains how to interpret an operation field for an
authentication test.

When using {\cpsa} for protocol design, focus on authentication tests.
For each iteration of the design, search for the most informative
unexpected intermediate skeleton.  That skeleton is likely to suggest
a  missing origination assumption, or the redesign of some message
term included in the operation field.

There are situations in which origination assumptions are not
justified for initial segments of runs of a protocol, but are required
to show that complete runs of the protocol have certain expected
shapes.  In this case, the progressive refinement analysis technique
is used.  The initial segments of runs are analyzed with only the
origination assumptions justified initially.  A shape associated with
a partial run may justify additional origination assumptions.  For
example, a strand in a partial run may send its next message only
after a trust decision is made, and the implication of the decision is
that the strand infers that some key is uncompromised.  In this
case, the shape with the new origination assumption and additional
regular behavior is supplied as input to {\cpsa}, thereby refining the
original problem.

As there is no guarantee that {\cpsa} is bug free, you may come upon
input that causes non-termination.  As a result, whenever you run the
program unattended, you should limit its memory usage.  To get output
that can be visualized, specify a step count and/or a strand bound so
that {\cpsa} has the chance to abort the run in a fashion that
generates graphable output.  Of course, sending us input that causes
erroneous behavior will help us improve {\cpsa}.

\section{Formula Extraction}\label{sec:formulas}

The \texttt{cpsasas} program extracts a formula in the language of
order-sorted first-order logic for each problem and its shapes from a
{\cpsa} run. The formula is called a shape analysis
sentence~\cite{Ramsdell12}. The formula is satisfied in all realized
skeletons when {\cpsa} finds all the shapes for the problem.  The
details of formula extraction are presented in Appendix~B of The
{\cpsa} Specification~\cite{cpsaspec09}.

\section{Annotations}\label{sec:annotations}

The \texttt{cpsaannotations} program uses protocol annotations to
annotate shapes and generate protocol soundness obligations for use
with the rely-guarantee method of trust
management~\cite{GuttmanEtAl04,RamsdellEtAl09}.  The language of
formulas is order-sorted first-order logic extended with a modal says
operator.  Formula terms may include function symbols that are not
part of a protocol's message signature.  The syntax of a formula is
specified in the {\cpsa} Overview~\cite{cpsaoverview09}.

\section{Parameter Analysis}\label{sec:parameters}

The parameters of a role are the atoms that are not acquired by the
role's trace, but must be available before a complete execution of the
trace is possible. The \texttt{cpsaparameters} program computes sets
of sets of parameters consistent with the role. If the expected
parameter set is not a member, a specification error is indicated.
The details of parameter analysis are presented in Appendix~A of The
{\cpsa} Specification~\cite{cpsaspec09}.

\section*{Acknowledgement}

Jonathan K.\@ Millen provided valuable feedback as our first {\cpsa}
user and on a draft of this document.

\appendix

\section{BCA Syntax Reference}\label{sec:syntax reference}

The complete syntax for the analyzer using the Basic Crypto Algebra is
shown in Tables~\ref{tab:syntax} and~\ref{tab:goal}.  The start
grammar symbol is \textsc{file}, and the terminal grammar symbols
are: \textsc{(, ), symbol, string, integer,} and the constants set in
typewriter font.  The language used for stating security goals is in
Table~\ref{tab:goal}.  The goal language adds the sort
symbol \texttt{node}.

The \textsc{alist}, \textsc{prot-alist}, \textsc{role-alist},
\textsc{skel-alist}, and \textsc{goal-alist} productions are
Lisp style association lists, that is, lists of key-value pairs, where
every key is a symbol.  Key-value pairs with unrecognized keys are
ignored, and are available for use by other tools.  On output,
unrecognized key-value pairs are preserved when printing protocols,
but elided when printing skeletons.

\begin{table}
\newcommand{\sym}[1]{\textup{\texttt{#1}}}
\begin{center}\scshape
\begin{tabular}{rcl}
file&$\leftarrow$&herald?~form+
\\herald&$\leftarrow$&
(\sym{herald}~title~alist)
\\title&$\leftarrow$&$\mbox{symbol}\mid\mbox{string}$
\\form&$\leftarrow$&
$\mbox{comment}\mid\mbox{protocol}\mid\mbox{skeleton}\mid\mbox{goal}$
\\ comment&$\leftarrow$&
(\sym{comment}~\ldots)
\\ protocol&$\leftarrow$&
(\sym{defprotocol} id alg role+ prot-alist)
\\ id&$\leftarrow$&symbol
\\ alg&$\leftarrow$&symbol
\\ role&$\leftarrow$&
(\sym{defrole} id vars trace role-alist)
\\ vars&$\leftarrow$&
(\sym{vars} decl$\ast$)
\\ decl&$\leftarrow$&
(id+ sort)
\\ sort&$\leftarrow$&
$\sym{text} \mid \sym{data} \mid \sym{name} \mid \sym{skey}
\mid \sym{akey}\mid\sym{mesg}$
\\ trace&$\leftarrow$&(\sym{trace} event+)
\\ event&$\leftarrow$&
$(\sym{send}\mbox{ term})\mid(\sym{recv}\mbox{ term})$
\\ term&$\leftarrow$&
$\mbox{id}\mid(\sym{pubk}\mbox{ id})
\mid(\sym{privk}\mbox{ id})
\mid(\sym{invk}\mbox{ id})$
\\ &$\mid$&$(\sym{pubk}\mbox{ id}\mbox{ string})
\mid(\sym{privk}\mbox{ id}\mbox{ string})$
\\ &$\mid$&$(\sym{ltk}\mbox{ id id})\mid\mbox{string}\mid(\sym{cat}\mbox{ term+})$
\\ &$\mid$&$(\sym{enc}\mbox{ term+ term})\mid(\sym{hash}\mbox{ term+})$
\\ role-alist&$\leftarrow$&$
(\sym{non-orig}\mbox{ ht-term}\ast)\mbox{ role-alist}$
\\ &$\mid$&$(\sym{pen-non-orig}\mbox{ ht-term}\ast)\mbox{ role-alist}$
\\ &$\mid$&$(\sym{uniq-orig}\mbox{ term}\ast)\mbox{ role-alist}\mid\ldots$
\\ ht-term&$\leftarrow$&term${}\mid{}$(integer term)
\\ prot-alist&$\leftarrow$&$\ldots$
\\ skeleton&$\leftarrow$&
(\sym{defskeleton} id vars
\\ &&\qquad strand+ skel-alist)
\\ strand&$\leftarrow$&
(\sym{defstrand} id integer maplet$\ast$)
\\ &$\mid$&(\sym{deflistener} term)
\\ maplet&$\leftarrow$&
(term term)
\\ skel-alist&$\leftarrow$&$(\sym{non-orig}\mbox{ term}\ast)\mbox{ skel-alist}$
\\ &$\mid$&$(\sym{pen-non-orig}\mbox{ term}\ast)\mbox{ skel-alist}$
\\ &$\mid$&$(\sym{uniq-orig}\mbox{ term}\ast)\mbox{ skel-alist}$
\\ &$\mid$&$(\sym{precedes}\mbox{ node-pair}\ast)\mbox{ skel-alist}\mid\ldots$
\\ node-pair&$\leftarrow$&
(node node)
\\ node&$\leftarrow$&
(integer integer)
\end{tabular}
\end{center}
\caption{{\cpsa} Syntax}\label{tab:syntax}
\end{table}

\begin{table}
\newcommand{\sym}[1]{\textup{\texttt{#1}}}
\begin{center}\scshape
\begin{tabular}{rcl}
goal&$\leftarrow$&
(\sym{defgoal} id sentence+ goal-alist)
\\ sentence&$\leftarrow$&(\sym{forall} (decl$\ast$) implication)
\\ implication&$\leftarrow$&
(\sym{implies} \mbox{antecedent} \mbox{conclusion})
\\ conclusion&$\leftarrow$&(\sym{false})
   $\mid$ existential $\mid$ (\sym{or} existential$\ast)$
\\ existential&$\leftarrow$&(\sym{exists}
(decl$\ast$) antecedent) $\mid$ antecedent
\\ antecedent&$\leftarrow$& atomic$\ast$
\\ atomic&$\leftarrow$&(\sym{p} string integer node)
\\ &$\mid$&(\sym{p} string string node term)
\\ &$\mid$&(\sym{str-prec} node node) $\mid$ (\sym{prec} node node)
\\ &$\mid$&(\sym{non} term) $\mid$ (\sym{pnon} term) $\mid$ (\sym{uniq} term)
\\ &$\mid$&(\sym{uniq-at} term node) $\mid$ (= term term)
\\ node&$\leftarrow$& id
\end{tabular}
\end{center}
\caption{{\cpsa} Syntax Continued}\label{tab:goal}
\end{table}

The contents of a file can be interpreted as a sequence of
S-expressions.  The S-expressions used are restricted so that most
dialects of Lisp can read them, and characters within symbols and
strings never need quoting.  Every list is proper.  An S-expression
atom is either a \textsc{symbol}, an \textsc{integer}, or
a \textsc{string}.  The characters that make up a symbol are the
letters, the digits, and the special characters in
``\verb|-*/<=>!?:$%_&~^+|''.  A symbol may not begin with a digit or a
sign followed by a digit.  The characters that make up a string are
the printing characters omitting double quote and backslash.  Double
quotes delimit a string.  A comment\index{comments} begins with a
semicolon, or is an S-expression list at top-level that starts with
the \texttt{comment} symbol.


\section{Omitted Executions}\label{sec:omitted executions}

When given some initial behavior, the {\cpsa} program discovers what
shapes are compatible with it.  Early work claimed that the {\cpsa}
algorithm would generate every shape in a finite number of
steps~\cite{DoghmiGuttmanThayer07,Guttman11}.  Further theoretical
work showed classes of executions that are not found by the algorithm,
however it also showed that every omitted execution requires an
unnatural interpretation of a protocol's roles.  Hence the algorithm
is complete relative to natural role semantics.  We present an example
of a protocol that has a shape missed by {\cpsa}.

The protocol has an initiator role and a responder role.
\begin{center}
\includegraphics{cpsadiagrams-6.mps}\hfil
\includegraphics{cpsadiagrams-7.mps}
\end{center}

Observe that a natural interpretation of~$N_2$ in the responder role
is that a binding for it is known when an instance of the role begins
execution.

Consider a scenario in which there is an instance of the initiator
role where~$N$ is freshly generated and~$K^{-1}$ is uncompromised.
{\cpsa} will find no shapes compatible with the scenario, however what
follows is.
\begin{center}
\includegraphics{cpsadiagrams-8.mps}
\end{center}
Yet this shape requires that the message for~$N_2$ in an instance the
responder role be obtained from its first message reception---an
unnatural interpretation of the responder role.

\bibliography{cpsa}
\bibliographystyle{plain}

\printindex

\end{document}
