\documentclass[12pt]{article}

% Much of this is taken from Deducing Security Goals From Shape
% Analysis Sentences II

\usepackage{url}
\usepackage[matrix,arrow,curve]{xy}
% \usepackage{hyperref}
\usepackage{url}

\usepackage{amssymb}
\usepackage{amsmath}
\usepackage{amsthm}
\usepackage{color}

\newtheorem{thm}{Theorem}
\newcommand{\cpsa}{\textsc{cpsa}}
\newcommand{\pvs}{\textsc{pvs}}
\newcommand{\cn}[1]{\ensuremath{\operatorname{\mathsf{#1}}}}
\newcommand{\dom}[1]{\ensuremath{\operatorname{\mathbf{#1}}}}
\newcommand{\fn}[1]{\ensuremath{\operatorname{\mathit{#1}}}}
\newcommand{\sdom}{\fn{dom}}
\newcommand{\sran}{\fn{ran}}
\newcommand{\vars}{\fn{Vars}}
\newcommand{\lvars}{\fn{lvars}}
\newcommand{\flvars}{\fn{flvars}}
\newcommand{\seq}[1]{\ensuremath{\langle#1\rangle}}
\newcommand{\enc}[2]{\ensuremath{\{\!|#1|\!\}_{#2}}}
\newcommand{\invk}[1]{{#1}^{-1}}
\newcommand{\inbnd}{\mathord -}
\newcommand{\outbnd}{\mathord +}
\newcommand{\neutral}{\ast\ast\ast}
\newcommand{\alga}{\ast\ast\ast}
\newcommand{\typ}{\mathbin:}
\newcommand{\srt}[1]{\ensuremath{\mathsf{#1}}}
\newcommand{\nat}{\ensuremath{\mathbb{N}}}
\newcommand{\all}[1]{\mathop{\forall#1\mathpunct.}}
\newcommand{\some}[1]{\mathop{\exists#1\mathpunct.}}
\newcommand{\prefix}[2]{#1\dagger#2}
\newcommand{\defd}{\mathord\downarrow}
\newcommand{\inst}{\star}
\newcommand{\init}{\fn{init}}
\newcommand{\resp}{\fn{resp}}
\newcommand{\form}{\mathcal{K}}
\newcommand{\sent}{\mathcal{S}}
\newcommand{\lang}{\mathcal{L}}
\newcommand{\tr}{\ensuremath{{\pm\msg\,}^+}}
\newcommand{\rl}{\fn{rl}}
\newcommand{\skel}{\mathbb{A}}
\newcommand{\skl}{\mathsf{k}}
\newcommand{\nodes}{\ensuremath{\mathcal{N}}}
\newcommand{\ztrands}{\ensuremath{\mathcal{Z}}}
\newcommand{\evt}{\fn{evt}}
%  This appears not to be used.
%  (JDG)
% \newcommand{\orig}{\mathcal{O}}
\newcommand{\length}[1]{\ensuremath{|#1|}}
\newcommand{\bun}{\ensuremath{\mathcal{B}}}
\newcommand{\insta}{\mathsf{i}}
\newcommand{\insts}{\mathcal{I}}
\newcommand{\role}{\mathsf{r}}
\newcommand{\pow}[1]{\wp(#1)}
\newcommand{\sembrack}[1]{[\![#1]\!]}

\newcommand{\morph}{\ensuremath{\mathbin{\cdot\joinrel\!\!\rightarrow}}}
\newcommand{\orig}{\ensuremath{\cn{uniq-at}}}

%  First:  Some alterations to the macros:

%

\newcommand{\alg}[1]{\ensuremath{\mathsf{#1}}}
\newcommand{\msg}{\alg{Alg}}
\newcommand{\atm}{\alg{BV}}

\newcommand{\atom}{basic value}
\newcommand{\atomic}{basic}
\newcommand{\Indefart}{A}
\newcommand{\indefart}{a}

\newcommand{\trace}{\fn{tr}}
\newcommand{\ssp}{\Sigma}

\newcommand{\strands}{\sdom}
\newcommand{\str}{\fn{str}}
\newcommand{\fac}{\fn{fac}}

\newcommand{\roleuo}{\fn{rl\_uniq}}
\newcommand{\roleno}{\fn{rl\_non}}

\newcommand{\run}{\fn{Bnd}}
\newcommand{\tracec}{\fn{trc}}
\newcommand{\tracei}{\fn{tri}}
\newcommand{\nonatomic}{{message}}
\newcommand{\mess}{\fn{msg}}

\newcommand{\skelB}{\mathbb{B}}
\newcommand{\skelC}{\mathbb{C}}
\newcommand{\skeleton}{\fn{sk}}
\newcommand{\skeletons}{\fn{skels}}
\newcommand{\Uni}{\mathcal{D}}

\newcommand{\sat}[3]{#1\models_{#2} #3}
\newcommand{\unsat}[3]{#1\not\models_{#2} #3}
\newcommand{\actuals}{\ensuremath{\fn{actl}}}
\newcommand{\cf}{\ensuremath{\fn{cf}}}
\newcommand{\hc}{\ensuremath{\fn{hc}}}
\newcommand{\hcf}{\ensuremath{\fn{hcform}}}

\newcounter{running}[section]
\newenvironment{renumerate}{\begin{enumerate}%
\setcounter{enumi}{\value{running}}}%
{\setcounter{running}{\value{enumi}}\end{enumerate}}

\newtheorem{definition}{Definition}
\newtheorem{lemma}{Lemma}

\newcommand{\instsJ}{\mathcal{J}}

\newcommand{\Adv}{\ensuremath{\mathsf{Adv}}}

%\institute{The MITRE Corporation}

\title{Strand Spaces with Rules}
\author{Joshua D.\ Guttman\and John D.\ Ramsdell}

\begin{document}
\maketitle

\section{Introduction}
This paper describes the formalism on which {\cpsa4} is built.
Section~\ref{sec:strand spaces} describes an implementation-oriented
specification of strand spaces.  It is based on the presentation for
{\cpsa2} in~\cite{cpsaspec09}, but modified so as to reflect
changes required by {\cpsa4}.  In particular, bundles and skeletons
have been modified so as to include facts.

Section~\ref{sec:rule language} presents the strand spaces rule
language that is used for rewriting.  Protocols have been redefined
so that they include both roles and rules.

\paragraph{Notation.}

A finite sequence~$f$ is a function from an initial segment of the
natural numbers.  The length of~$f$ is~$\length{f}$, and $f=\seq{f(0),\ldots,
  f(n-1)}$ for $n=\length{f}$.  When $f$ is a sequence, we will write
$g\colon\length{f}\rightarrow S$ to mean that $\sdom(g)=\{i\colon0\le
i<\length{f}\}$ and $\sran(g)=S$.  Thus, in effect we are regarding
each natural number $i$ as being the set of natural numbers smaller
than it, like the ordinals of ZF set theory.

If~$S$ is a set, then~$S^\ast$ is the set of finite sequences
over~$S$, and~$S^+$ is the non-empty finite sequences over~$S$.
\iffalse
The concatenation of sequences~$f_0$ and~$f_1$ is~$f_0\append f_1$.
\fi
The prefix of sequence~$f$ of length~$n$ is~$\prefix{f}{n}$.
For partial function~$g$, $g(x)\defd$ asserts $g$ is defined at~$x$.

\section{Implementation-Oriented Strand Spaces}\label{sec:strand spaces}

This section describes the formalism on which {\cpsa} is based.  The
parameters to this strand space theory are:
\begin{enumerate}
  \item a set of messages \msg.  The set of messages~{\msg} is the
  carrier set (or domain) of a term algebra.
  \item a set of {\atom}s $\atm\subset\msg$.  Keys and nonces are
  examples of {\atom}s.
  \item a \emph{carried by} relation
  ${\sqsubseteq}\subseteq\msg\times\msg$, also known sometimes as the
  \emph{ingredient} relation.  Intuitively, a message~$t_0$ is carried
  by~$t_1$, written $t_0\sqsubseteq t_1$, if it is possible to
  extract~$t_0$ from~$t_1$ for someone who knows the relevant
  decryption keys.
  \item a set of application specific predicate symbols $P$.
\end{enumerate}

\paragraph{Example Message Algebra.}
The signature of one possible order-sorted~\cite{GoguenMeseguer92}
message algebra is in Figure~\ref{fig:signature}.  The algebra is the
simplification of the {\cpsa} message algebra used by the examples in
this paper.  In an order-sorted algebra, each variable~$x$ has a
unique sort~$S$.  The \emph{declaration} of~$x$ is $x\typ S$.

The algebra of interest is the order-sorted quotient term algebra
generated by a set of declarations~$X$.  The message algebra~$\msg_X$
is the carrier set for sort~\srt{M}.  The set of {\atom}s~$\atm_X$ is
the union of the carrier sets for sorts \srt{A}, \srt{S}, and \srt{D}.
We write $t\typ S$ to say that term~$t$ is in the carrier set of
sort~$S$.

\begin{figure}
$$\begin{array}{ll@{{}\colon{}}ll}
\mbox{Sorts:}&
\multicolumn{3}{l}{\mbox{\srt{M}, \srt{A}, \srt{S}, \srt{D}}}\\
\mbox{Subsorts:}&
\multicolumn{3}{l}{\mbox{$\srt{A}<\srt{M}$, $\srt{S}<\srt{M}$,
    $\srt{D}<\srt{M}$}}\\
\mbox{Operations:}&(\cdot,\cdot)&\srt{M}\times\srt{M}\to\srt{M}
&\mbox{Pairing}\\
&\enc{\cdot}{(\cdot)}&\srt{M}\times\srt{A}\to\srt{M}
&\mbox{Asymmetric encryption}\\
&\enc{\cdot}{(\cdot)}&\srt{M}\times\srt{S}\to\srt{M}
&\mbox{Symmetric encryption}\\
&(\cdot)^{-1}&\srt{A}\to\srt{A}
&\mbox{Asymmetric key inverse}\\
&\tau_0,\tau_1,\ldots&\srt{M}
&\mbox{Tag constants}\\
\mbox{Equations:}&\multicolumn{3}{l}{(x^{-1})^{-1}=x\mbox{ for $x\typ\srt{A}$}}
\end{array}$$
\caption{Simple Crypto Algebra Signature}\label{fig:signature}
\end{figure}

A variable has no intrinsic sort associated with it.  Variables occur
in the context of a set of declarations, and a declaration for the
variable specifies its sort.  The set of variables that occur in
term~$t$ is $\vars(t)$.  A variable declared to be of sort~\srt{M} is
called a \emph{{\nonatomic} variable}.

A message~$t_0$ is \emph{carried by}~$t_1$, written $t_0\sqsubseteq
t_1$, if~$t_0$ can be derived from~$t_1$ given the right set of keys.
That is:  $\sqsubseteq$ is the smallest reflexive, transitive relation
such that
%
$$t_0\sqsubseteq (t_0, t_1),\quad
t_1\sqsubseteq (t_0, t_1),\quad\mbox{ and }\quad
t_0\sqsubseteq\enc{t_0}{t_1}.$$

\paragraph{Strand Spaces.}
A run of a protocol is viewed as an exchange of messages by a finite
set of local sessions of the protocol.  Each local session is called a
\emph{strand}.  The behavior of a strand, its \emph{trace}, is a
finite non-empty sequence of \emph{events}.  An \emph{event} is either
a \emph{message transmission} or a \emph{message reception}.  An event
transmitting $t\in\msg$ is written as~$\outbnd t$; and an event
receiving message~$t$ is written as~$\inbnd t$.  If $e=\pm t$ is an
event, then $\mess(e)=t$.  The set of traces over $\msg$ is
$(\pm\msg)^+$.

\begin{renumerate}
  \item A message $t$ \emph{originates} in trace~$c$ at index~$i$ iff
  $c(i)=+t_1$, $t\sqsubseteq t_1$, and for all $j<i$,
  $t\not\sqsubseteq\mess(c(j))$.
  \item A message $t$ is \emph{acquired} in trace~$c$ at index~$i$ iff
  $c(i)=-t_1$, $t\sqsubseteq t_1$, and for all
  $j<i$, $t$ does not occur in $\mess(c(j))$.
  \item A variable $x$ \emph{appears} in trace~$c$ at index~$i$ iff
    $x$ occurs in $\mess(c(i))$, and for all $j<i$, $x$ does not occur
    in $\mess(c(j))$.
  \item For algebra homomorphism~$\sigma$,
%
  \[\sigma\circ\seq{\pm t_0,\ldots,\pm t_{n-1}}=
  \seq{\pm \sigma(t_0),\ldots,\pm \sigma(t_{n-1})},\]
%
  so that $\prefix{(\sigma\circ
    c)}{h}=\sigma\circ(\prefix{c}{h})=\sigma\circ\prefix{c}{h}$.
  \item For traces~$c_1$, $c_2$, $c_1\inst c_2=\sigma$ if
    $\sigma(\prefix{c_2}{\length{c_1}})=c_1$, and the domain
    of~$\sigma$ is the variables that occur in
    $\prefix{c_2}{\length{c_1}}$.  When $c_1\inst c_2$ is defined,
    $c_1$ is said to be an \emph{instance} of $c_2$.
  \item A \emph{strand space} is a set $\Sigma$ of values, which we
  will call \emph{the strands of $\Sigma$}, equipped with a
  \emph{trace-of} operator
  $\trace\colon\Sigma\rightarrow\tr$.
\end{renumerate}
%
To avoid excess notation, we generally suppress $\trace$ and regard
$\ssp$ as the strand space.  We write $\sdom(\ssp)$ for the underlying
set of strands, and $\ssp(s)$ for $\trace(s)$ when $s\in\sdom(\ssp)$.

The set of strands $\sdom(\ssp)$ that {\cpsa} uses are of a specific
kind.  They are finite initial segments of the natural numbers, so
that $\ssp$ is in fact a sequence of traces.

%
Message events occur at \emph{nodes} in a strand space.  For each
strand~$s$, there is a node for every event in~$\ssp(s)$.
\begin{renumerate}
  \item The \emph{nodes} of strand space $\ssp$ are
%
  $$ \nodes(\ssp)=\{(s,i)\mid
  s\in\strands(\ssp), i < \length{\ssp(s)}\}
  $$
  %
  and the event at a node is $\evt_{\ssp}(s,i)=\ssp(s)(i)$.
  \item The \emph{message at} a node $n$, written $\mess_{\ssp}(n)$,
    is $\mess(\evt_{\ssp}(n))$.
  \item The \emph{strand succession relation}~$\Rightarrow$ is defined
  by
%
  $$(s,i)\Rightarrow(s,i+1)\mbox{ iff } s\in\strands(\ssp)\mbox{ and }
  i<\length{\ssp(s)}-1.$$
%
  \item {\Indefart} {\atom}~$t$ is \emph{non-originating} \emph{in a
    strand space}~$\ssp$, written $\fn{non}(\ssp,t)$, iff it
    originates on no strand in~$\ssp$ and each variable in $t$ occurs
    in $\ssp$.
  \item {\Indefart} {\atom}~$t$ \emph{uniquely originates at node} $n$
    \emph{in strand space}~$\ssp$, written $\fn{uniq}(\ssp,t,n)$,
    iff $t$ originates at index $i$ in $s\in\sdom(\ssp)$, $n=(s,i)$,
    and $t$ originates on no other strand in $\sdom(\ssp)$.
  \item An atomic formula $p(t_1,\ldots,t_m)$ is a
    \emph{fact of} $\ssp$ iff
    \begin{enumerate}
    \item $p\in P$ is a predicate symbol,
    \item $t_i\in\msg\cup\sdom(\ssp)$, and
    \item each variable that occurs in $t_i$ occurs in $\ssp$.
    \end{enumerate}
\end{renumerate}

\begin{definition}[Bundle]\label{def:bundle} Let $\bun=(\ssp,\to,\omega)$, where
  $\to\subseteq\nodes(\ssp)\times\nodes(\ssp)$ and $\omega$ is a set
  of atomic formulas.  Relation $\to$ is called the
  \emph{communication relation}.  $\bun$ is a \emph{bundle} iff:
  \begin{enumerate}
    \item $n_0\to n_1$ implies $\evt_{\ssp}(n_0)=\outbnd t$
    and~$\evt_{\ssp}(n_1)=\inbnd t$ for some message~$t$;
    \item $\evt_{\ssp}(n_1)=\inbnd t$ implies there exists a unique
      $n_0$ such that $n_0\to n_1$;
    \item $\nodes(\ssp),\to\cup\Rightarrow$ forms a well-founded
      directed graph; and
    \item for all $f\in\omega$, $f$ is a fact of $\ssp$.
  \end{enumerate}
  Let $\prec_{\bun}=(\to\cup\Rightarrow)^+$, the transitive closure of
  the edges.
%
\end{definition}

%
The transitive closure $\prec_{\bun}$ is an irreflexive relation on
$\nodes(\ssp)$.  This transitive, irreflexive relation specifies the
causal ordering of nodes in a bundle.  A transitive irreflexive binary
relation is also called a strict (partial) order.

Since {\cpsa} manipulates only finite strand spaces, and acyclicity is
equivalent to well-foundedness for finite graphs, {\cpsa} just checks
acyclicity.  Thus, in $\bun=(\ssp,\to,\omega)$, the nodes form the vertices
of a directed acyclic graph, whose edges represent communications
$\rightarrow$ and strand succession $\Rightarrow$ in~$\ssp$.
%
\begin{lemma}
  Let $\bun=(\ssp,\to,\omega)$ be a bundle.  Then $\prec_{\bun}$ is a
  well-founded strict partial order.

  If $S\subseteq\nodes(\ssp)$ is non-empty, then $S$ has
  $\prec_{\bun}$-minimal members.
\end{lemma}

\begin{renumerate}
  \item The strand space of bundle~{\bun} is written~$\str(\bun)$, its
    communication relation is~$\to_\bun$, and the facts are
    $\fac(\bun)$, so $\bun=(\str(\bun),\to_\bun,\fac(\bun))$.
\end{renumerate}
%

\paragraph{Runs of Bare Protocols.}
%
In a run of a bare protocol, each strand is an instance of a role of
that protocol.  We view adversarial strands as constrained by roles,
just like compliant, non-adversarial strands.  Recall that when $f$ is
a sequence, $g\colon\length{f}\rightarrow S$ means that
$\sdom(g)=\{i\colon0\le i<\length{f}\}$.
%
\begin{definition}[Role, Bare Protocol]\label{def:role bare protocol}
  ~
  \begin{enumerate}
    \item A \emph{role} is of the form $\role_X(c)$, where
    \begin{description}
      \item[$X$] is the \emph{parameters} of the role, the
        declarations used to generate its algebra.
      \item[$c\in\tr_X$] is the trace of the role;
    \end{description}
%
    satisfying the following property:
%
%  A proto-role $\role(c,\nu,\upsilon)$ is a \emph{role} iff
%
    \begin{enumerate}
      \item every {\nonatomic} variable is acquired on $c$, i.e.~its
      first occurrence is in a reception event of $c$;
    \end{enumerate}
    \item The \emph{listener role} is
      $\fn{lsn}=r_{\{x:\srt{M}\}}(\seq{\inbnd x, \outbnd x})$.
    \item A \emph{bare protocol} is a set of roles that includes the
      listener role.
  \end{enumerate}
\end{definition}
%
Notice that in one role, variable~$k$ may be declared to be of
sort~\srt{A}, and in another, of sort~\srt{S}.
%
\begin{definition}\label{def:instance}
  A strand $s\in\ssp$ is an \emph{instance} of a role $\role_Y(c)$ in
  strand space $\ssp$ over $\msg_X$ iff the function
  $\fn{inst}(\ssp,s,\role_Y(c))$ is defined, and
  $\fn{inst}(\ssp,s,\role_Y(c)) = \sigma$ when
  \begin{enumerate}
    \item $\length{\ssp(s)}\leq\length{c}$;
    \item $\sigma\in\msg_Y\to\msg_X$; and
    \item $\sigma = \ssp(s)\inst c$.
  \end{enumerate}
\end{definition}
%
We use the variables $\rho,\rho',\rho_i$, etc., to range over roles.

\begin{figure}
  $$\begin{array}{r@{{}={}}ll}
    \fn{create}&\seq{\outbnd x}&
    \mbox{$x\typ\srt{A}$ or $x\typ\srt{S}$ or $x\typ\srt{D}$}\\
    \fn{pair}&
    \seq{\inbnd x,\inbnd y,\outbnd (x,y)}&
    x,y\typ\srt{M}\\
    \fn{sep}&
    \seq{\inbnd (x,y),\outbnd x,\outbnd y}&
    x,y\typ\srt{M}\\
    \fn{aenc}&
    \seq{\inbnd x,\inbnd k,\outbnd \enc{x}{k}}&
    \mbox{$x\typ\srt{M}$ and $k\typ\srt{A}$}\\
    \fn{adec}&
    \seq{\inbnd \enc{x}{k},\inbnd\invk{k},\outbnd x}&
    \mbox{$x\typ\srt{M}$ and $k\typ\srt{A}$}\\
    \fn{senc}&
    \seq{\inbnd x,\inbnd k,\outbnd \enc{x}{k}}&
    \mbox{$x\typ\srt{M}$ and $k\typ\srt{S}$}\\
    \fn{sdec}&
    \seq{\inbnd \enc{x}{k},\inbnd k,\outbnd x}&
    \mbox{$x\typ\srt{M}$ and $k\typ\srt{S}$}
  \end{array}$$
  \caption{Adversary Roles}\label{fig:adversary}
\end{figure}

\paragraph{Adversary Model.}
The roles of adversarial behavior are in Figure~\ref{fig:adversary}.
There are three \fn{create} roles, one for each basic sort.  In fact,
the defining characteristic of {\indefart} {\atom} is that it denotes
the set of messages the adversary can create out of thin air,
consistent with origination assumptions.

The adversary roles form the bare protocol $\Adv$.  We are always
interested in bare protocols that contain these roles.  Thus, given a set
$\Pi$ of roles that represent the legitimate, compliant behaviors of
principals in some (real-world) bare protocol, we will refer to the roles
of $\Pi$ as the \emph{regular} behaviors, and to strands that
instantiate them as \emph{regular strands}.  We will be interested in
the bundles that are executions of the ``penetrated'' bare protocol
$\Pi^+=\Pi\cup\Adv$.
%
\begin{renumerate}
%
  \item A bundle~$\bun=(\ssp,\to,\omega)$ is a \emph{run of bare
    protocol} $\Pi$ iff, for every $s\in\strands(\ssp)$, there is a
    role $\rho\in\Pi^+=\Pi\cup\Adv$ such that
    $\fn{inst}(\ssp,s,\rho)\defd$.
%
  \item $\run(\Pi)$ is the set of bundles that are runs of bare
    protocol~$\Pi$.

  \item For $\bun\in\run(\Pi)$, $\varrho\in\sdom(\str(\bun))\to\Pi^+$ is a
    \emph{role assignment} if for every $s\in\sdom(\str(\bun))$,
    $\fn{inst}(\str(\bun),s,\varrho(s))\defd$.

\end{renumerate}
%
\begin{lemma}
  If $\bun\in\run(\Pi)$ and $x$ is a {\nonatomic} variable, then
  $x\not\in\vars(\bun)$.
\end{lemma}
%
Ensuring this is the purpose of requiring that in each role, every
{\nonatomic} variable in the trace is acquired.

\paragraph{Skeletons.}

A \emph{skeleton} represents part of the regular behavior in a set of
bundles.  It consists of a strand space, a partial ordering on the
nodes, assumptions about uncompromised keys and about freshly
generated {\atom}s, and a set of facts.

\begin{definition}\label{def:skeleton}
  $\skel=\skl_X(\Pi,\ssp,\prec,\nu,\upsilon,\omega)$ is a
  \emph{skeleton of} bare protocol $\Pi$ over declarations $X$ iff:
%
  \begin{enumerate}
    \item A variable is declared in~$X$ iff it occurs in $\ssp$.
    \item For all $s<\length{\ssp}$, there is some $\rho\in\Pi$ such
      that $\fn{inst}(\ssp,s,\rho)\defd$;
    \item $\prec$ is a strict (partial) order on $\nodes(\ssp)$;
    \item for all $t\in\nu$, (i) $t\in\atm_X$; and (ii) for all
    $n\in\nodes(\ssp)$, $t\not\sqsubseteq\mess(n)$;
    \item for all $(t,n)\in\upsilon$, (i) $t\in\atm_X$; (ii) for some
      $n=(s,i)$, $t$ originates in $\ssp(s)$ at~$i$; and (iii) $t$
      originates at no other node in $\nodes(\ssp)$.
    \item for all $f\in\omega$, $f$ is a fact of $\ssp$.
  \end{enumerate}
%
\end{definition}

We regard skeletons as giving us \emph{partial information} about a
set of bundles; these are the executions that it is compatible with.
This notion of compatibility is determined by the the notion of a
homomorphism, or information-preserving map between skeletons.

\begin{definition}[Homomorphism]\label{def:homomorphism}
  Let $\skel_0=\skl_X(\Pi,\ssp_0,\prec_0,\nu_0,\upsilon_0,\omega_0)$ and
  $\skel_1=$ $\skl_Y(\Pi,\ssp_1,\prec_1,\nu_1,\upsilon_1,\omega_1)$ be
  skeletons of bare protocol $\Pi$.

  $H=(\phi,\sigma)$ is a \emph{skeleton homomorphism}, written
  $H\colon \skel_0\morph \skel_1$, if~$\phi$ and~$\sigma$ are maps
  with the following properties:
  \begin{enumerate}
    \item $\phi\colon\strands(\ssp_0)\to\strands(\ssp_1)$ maps strands
    of~$\skel_0$ into those of~$\skel_1$.  We require
    $\length{\ssp_0(s)}\le\length{\ssp_1(\phi(s))}$, and we regard
    $\phi$ also as mapping nodes by the rule
    $\phi((s,i))=(\phi(s),i)$;
    \item $\sigma\colon\msg_X\to\msg_Y$ is a message algebra homomorphism;
    \item for all $n\in\nodes(\ssp_0)$,
    $\sigma(\evt_{\ssp_0}(n))=\evt_{\ssp_1}(\phi(n))$;
    \item $n_0\prec_0 n_1$ implies $\phi(n_0)\prec_1\phi(n_1)$;
    \item $\sigma(\nu_0)\subseteq \nu_1$;
    \item\label{item:orig:jdg} $(t,n)\in \upsilon_0$ implies
      $(\sigma(t), \phi(n))\in \upsilon_1$; and
    \item $p(t_1,\ldots,t_m)\in\omega_0$ implies
      $p(\delta(t_1),\ldots,\delta(t_m))\in\omega_1$, where
      \[\delta(t)=\left\{
      \begin{array}{ll}
        \sigma(t)&\mbox{if $t\in\msg$}\\
        \phi(t)&\mbox{if $t\in\sdom{\ssp}$.}
      \end{array}\right.
      \]

  \end{enumerate}
%
\end{definition}
%
Property~\ref{item:orig:jdg} says the node at which {\indefart}
{\atom} is declared to be uniquely originating is preserved by
homomorphisms.
%
\begin{definition}[Skeleton of Bundle]\label{def:homomorphism to bundle}
  ~
  \begin{enumerate}
    \item Suppose that $\bun$ is a bundle of bare protocol
      $\Pi^+=\Pi\cup\Adv$, and
      $\skel=\skl_X(\Pi,\ssp,\prec,\nu,\upsilon,\omega)$ is a $\Pi$-skeleton.
      $\skel$ is a skeleton of $\bun$, written
      $\skel\in\skeletons(\bun)$, iff the strands of $\bun$ can be
      permuted to form bundle $\overline{\bun}$ such that
      $\ssp$ is a prefix of $\str(\overline{\bun})$, and:
      \begin{enumerate}
      \item the strands not in $\ssp$ are adversary strands;
      \item ${\prec}={\prec_{\overline\bun}\cap(\nodes(\ssp)\times\nodes(\ssp))}$;
      \item $\nu=\{t\mid\fn{non}(\overline\bun,t)$ and each variable in~$t$
        occurs in $\ssp\}$;
      \item $\upsilon=\{(t,n)\mid\fn{uniq}(\overline\bun,t,n)
        \land n\in\nodes(\ssp)\}$; and
      \item $\omega=\{f\in\fac(\overline\bun)\mid
        \mbox{$f$ is a fact of $\ssp$}\}$.
    \end{enumerate}
    \item $\sembrack{\skel}=\{\bun\colon\some{\skelB}\some{H}
    \skelB\in\skeletons(\bun)\land H\colon \skel\morph\skelB\}$.

  \end{enumerate}

\end{definition}

\section{Strand Spaces Rule Language}\label{sec:rule language}

This section describes an order-sorted first-order language intimately
tied to strand spaces with protocols.  Given a bare protocol $\Pi$,
the strand spaces rule language is written $\lang(\Pi)$.  The
signature of $\lang(\Pi)$ includes the sorts, sort orderings, and the
function symbols of the protocol's message algebra signature, such as
the one in Figure~\ref{fig:signature}.  It also includes the strand
sort $\srt{Z}$, and no other additions.

The predicates in the signature of $\lang(\Pi)$ will now be described.
For each role $\role_X(c)\in\Pi$, there are height and parameter
predicates.  There are $\length{c}$ unary height predicates
$\Pi[\role_X(c),h]\typ\srt{Z}$ with $1\leq h \leq\length{c}$.
Relative to skeleton~$\skel$, $\Pi[\rho,h](z)$ asserts that strand~$z$
in~$\skel$ is an instance of~$\rho$ and has a height of at least~$h$.
For each parameter $x\typ S\in X$, there is a binary parameter
predicate $\Pi[\role_X(c),x]\typ\srt{Z}\times S$.  Relative to
skeleton~$\skel$, $\Pi[\rho,x](z,t)$ asserts that strand~$z$
in~$\skel$ is an instance of~$\rho$ in which~$x$ is instantiated
as~$t$.  If $x$ appears in $c$ and index $i$, the height of strand $z$
must be greater than~$i$.

For each base sort~$B$, there are unary predicates $\cn{non}\colon B$
and $\cn{uniq}\colon B$.  $\cn{non}(t)$ asserts~$t$ is non-originating
in~$\skel$ and $\cn{uniq}(t)$ asserts~$t$ uniquely originates in~$\skel$.

Let $m$ be the length of the longest role in~$\Pi$.  There are $m^2$
precedence predicates $\cn{prec}[i,j]:\srt{Z}\times\srt{Z}$ for $0\leq
i,j< m$.  $\cn{prec}[i,j](x,y)$ asserts that node $(x,i)$ is before
node $(y,j)$ in~$\skel$.  There are $3m$ origin predicates
$\cn{uniq-at}[i]:B\times\srt{Z}$, with $0\leq i< m$ and $B$ as before.
$\cn{uniq-at}[i](t,z)$ asserts that~$t$ uniquely originates in~$\skel$
at node $(z,i)$.  For each fact predicate symbol $p$,
$\cn{fact}[p](t_1,\ldots,t_m)$ asserts that $p(t_1,\ldots,t_m)$ is a
fact of $\skel$.  Finally, equality is binary.

To improve the readability of formulas to follow, we write
$\cn{prec}(x,i,y,j)$ for $\cn{prec}[i,j](x,y)$, $\cn{uniq-at}(t,z,i)$
for $\cn{uniq-at}[i](t,z)$, and $\cn{fact}(p,t_1,\ldots,t_m)$ for
$\cn{fact}[p](t_1,\ldots,t_m)$.

\paragraph{Semantics of Strand Space Rule Formulas.}

When formula~$\Phi$ is satisfied in skeleton~$\skel$ with order-sorted
variable assignment $\alpha$, we write $\skel,\alpha\models\Phi$.  For
$x:S\in X$, $\alpha(x)$ is in the carrier set of $\msg_X$ for
sort~$S$.  For $x:\srt{Z}$, $\alpha(x)\in\sdom(\ssp)$.  We
write~$\bar\alpha$ when~$\alpha$ is extended to terms in the obvious
way.  When sentence~$\Phi$ is satisfied in skeleton~$\skel$, we write
$\skel\models\Phi$.  The semantics of atomic formulas is given in
Figure~\ref{fig:semantics}.

\begin{figure}
  Let $\skel=\skl_X(\Pi,\ssp,\prec,\nu,\upsilon,\omega)$.
  \begin{itemize}
  \item $\skel,\alpha\models\Pi[\role_X(c),h](z)$ iff
    $\length{\ssp(s)}\geq h$ and
    $\prefix{\ssp(s)}{h}\inst c\defd$, where $s=\alpha(z)$.
  \item $\skel,\alpha\models P[\role_X(c),x](z,t)$ iff
    $\length{\ssp(s)}\geq h$,
    $\prefix{\ssp(s)}{h}\inst c=\sigma$, and
    $\sigma(x)=\bar\alpha(t)$,
    where $s=\alpha(z)$, $x$ appears in $c$ at index $i$, and $h=i+1$.
  \item $\skel,\alpha\models\cn{prec}(x,i,y,j)$ iff
    $(\alpha(x),i)\prec(\alpha(y),j)$.
  \item $\skel,\alpha\models\cn{non}(t)$ iff $\bar\alpha(t)\in\nu$.
  \item $\skel,\alpha\models\cn{uniq}(t)$ iff
    ($\bar\alpha(t),n)\in\upsilon$ for some node $n$.
  \item $\skel,\alpha\models\cn{uniq-at}(t,z,i)$ iff
    $(\bar\alpha(t),(\alpha(z),i))\in\upsilon$.
  \item $\skel,\alpha\models\cn{fact}(p,t_1,\ldots,t_m)$ iff
    $p(\bar\alpha(t_1),\ldots,\bar\alpha(t_m))\in\omega$.
  \item $\skel,\alpha\models y=z$ iff $\bar\alpha(y)=\bar\alpha(z)$.
  \end{itemize}
  \caption{Strand Space Rule Formula Semantics}\label{fig:semantics}
\end{figure}

\paragraph{Rule Syntax.}

Given a bare protocol $\Pi$, a rule is an order-sorted first-order
sentence in $\lang(\Pi)$ with a restricted syntax.  A rule is an
implication in which the antecedent is a conjunction of atomic
formulas, and the conclusion is a disjunction of possibly
existentially quantified conjunctions of atomic formulas.

\begin{definition}[Rule]\label{def:rule}
  A \emph{rule} is a sentence of the form
  \[\all{\vec{x}}\Phi\supset\bigvee_i\some{\vec{y}_i}\Psi_i,\mbox{
    where}\]
  \begin{enumerate}
  \item $\Phi$ and $\Psi_i$ are conjunctions of atomic formulas,
  \item each variable that occurs free in $\vee_i\some{\vec{y}_i}\Psi_i$,
    occurs in $\Phi$,
  \item each strand variable in $\vec{y}_i$ occurs in a height formula
    in $\Psi_i$, and
  \item each non-strand variable in $\vec{y}_i$ occurs in a parameter
    formula in $\Psi_i$.
  \end{enumerate}
\end{definition}

\begin{definition}[Protocol]\label{def:protocol}
  $\Upsilon=(\Pi,A)$ is a \emph{protocol} iff $\Pi$ is a bare protocol
  and $A$ is a set of rules in $\lang(\Pi)$.
\end{definition}

\begin{definition}[Run of Protocol]\label{def:run of protocol}
  Bundle $\bun$ is a \emph{run of protocol} $\Upsilon=(\Pi,A)$ iff
  $\bun$ is a run of bare protocol $\Pi$ and for all
  $\skel\in\skeletons(\bun)$ and rules $\Phi\in A$,
  $\skel\models\Phi$.
\end{definition}

The strand spaces rule language is a descendent of the security goal
language~\cite{Guttman14}.  They are very similar, but differ in several
ways.

\begin{enumerate}
\item Pairing and encryption is purposely omitted from the security
  goal language.
\item The security goal language has variables that range over nodes
  rather than strands as is the case for the strand spaces rule
  language.
\item Node variables that occur in the antecedent of a security goal
  sentence occur in a height predicate in the antecedent.
\item Non-node variables that occur in the antecedent of a sentence occur
  in a parameter predicate in the antecedent.
\item Facts are not part of the security goal language.
\end{enumerate}

\bibliography{cpsa}
\bibliographystyle{plain}

\end{document}
